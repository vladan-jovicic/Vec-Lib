%/!\ /!\ 
%
% PLEASE DO NOT EDIT THIS IF YOU CAME HERE BY MISTAKE !!!!
%

% RTFMN : https://tobi.oetiker.ch/lshort/lshort.pdf

\documentclass{article}
\usepackage{xspace}
\usepackage[utf8]{inputenc}
\usepackage[T1]{fontenc}
\usepackage[english]{babel}
\usepackage{amsmath}
\usepackage{amsthm}
\usepackage{graphicx}
\usepackage{url}
\usepackage{amssymb}
\usepackage{mathrsfs}
\usepackage{amsfonts}
\usepackage{multicol}
\usepackage{stmaryrd}
%\usepackage{algorithme}
\usepackage{tikz, pgf}
\usetikzlibrary{arrows,intersections}
\usepackage{libertine}
\usepackage[a4paper,left=2cm,right=2cm,top=2cm,bottom=2cm]{geometry}
\usepackage{dsfont}


\usepackage[linktocpage]{hyperref}

\setlength{\hoffset}{-18pt}         
\setlength{\oddsidemargin}{15pt} % Marge gauche sur pages impaires
\setlength{\evensidemargin}{15pt} % Marge gauche sur pages paires
\setlength{\marginparwidth}{0pt} % Largeur de note dans la marge
\setlength{\textwidth}{481pt} % Largeur de la zone de texte 
\setlength{\marginparsep}{7pt} % Séparation de la marge
\setlength{\topmargin}{0pt} % Pas de marge en haut
\setlength{\headheight}{13pt} % Haut de page
\setlength{\headsep}{10pt} % Entre le haut de page et le texte
\setlength{\footskip}{50pt} % Bas de page + séparation
\setlength{\textheight}{600pt} % Hauteur de la zone de texte 

%\setlength{\hoffset}{-18pt}         
%\setlength{\oddsidemargin}{15pt} % Marge gauche sur pages impaires
%\setlength{\evensidemargin}{15pt} % Marge gauche sur pages paires
%\setlength{\marginparwidth}{0pt} % Largeur de note dans la marge
%\setlength{\textwidth}{481pt} % Largeur de la zone de texte 
%\setlength{\marginparsep}{7pt} % Séparation de la marge
%\setlength{\topmargin}{0pt} % Pas de marge en haut
%\setlength{\headheight}{8pt} % Haut de page
%\setlength{\headsep}{0pt} % Entre le haut de page et le texte
%\setlength{\footskip}{15pt} % Bas de page + séparation
%\setlength{\textheight}{700pt} % Hauteur de la zone de texte 

%\newcommand{\ket}[1]{\ensuremath{|#1\rangle}\xspace}
%\newcommand{\bra}[1]{\ensuremath{\langle #1|}\xspace}

\newtheorem{thm}{Theorem}[section]
\newtheorem{prop}[thm]{Proposition}
\newtheorem{lem}[thm]{Lemma}
\newtheorem{cor}[thm]{Corollary}
\newtheorem{defi}[thm]{Definition}

\newcommand{\Thm}[3]{\begin{thm}[#1]\label{#2}#3\end{thm}}
\newcommand{\Ex}[3]{\begin{ex}[#1]\label{#2}#3\end{ex}}
\newcommand{\Def}[3]{\begin{defi}[#1]\label{#2}#3\end{defi}}
\newcommand{\Lem}[3]{\begin{lem}[#1]\label{#2}#3\end{lem}}
\newcommand{\Cor}[3]{\begin{cor}[#1]\label{#2}#3\end{cor}}
\newcommand{\Prop}[3]{\begin{prop}[#1]\label{#2}#3\end{prop}}

\newcommand{\hsp}{\hspace{20pt}}
\newcommand{\HRule}{\rule{\linewidth}{0.5mm}}
\newcommand{\R}{\mathbb{R}}
\newcommand{\N}{\mathbb{N}}
\newcommand{\K}{\mathbb{K}}
\newcommand{\Q}{\mathbb{Q}}
\newcommand{\C}{\mathcal{C}}
\newcommand{\brackets}[1]{\langle#1\rangle}

\newcommand{\ind}[1]{\mathds{1}_{#1}}


\title{About Integer programming to improve the polygonization step.}
\author{Quentin Guilmant}
\date{\today}

\begin{document} 
\maketitle
\section{Introduction}
The polygonization step aims to build a polygon given a set of point ad an input. It has to approximate the set by such a polygon in order to do the computation of the curves. If we have less point, the computation will be easier but may be it will be make some mistakes. The aim of this part is to compute the best set of point that minimize the cardinality and that satisfies a small distance to the convex hull of the point.

\section{Working on convex sets}

\section{Integer programming}
\subsection{Minimizing the number of vertices of the polygon.}
Suppose we have a distance $\delta$ such that each initial point of the set of point we have as an input must have a distance to the resulting polygon at most $\delta$.

We denote $S\subseteq \R^2$ the set of point we have as an input. We denote $d$ the usual euclidean distance. We denote for every $e\in S$ $x_e\in\{0,1\}$ is the variable for integer programming corresponding to $e$. The we want:
\[\forall e,f,f'\in S,x_f=x_{f'}=1, d(e, [f,f'])\leq\delta\]
Suppose we have such $e,f,f'$. Then $d(e,[f,f'])=\sqrt{\brackets{\vec{fe},\vec{fe}}-\brackets{\vec{fh},\vec{fh}}}$ with $\vec{fh}=\vec{ff'}\frac{\brackets{\vec{fe},\vec{ff'}}}{\|\vec{ff'}\|^2}$. Basically, $h$ is the orthogonal projection or $e$ on $[f,f']$.\\
$d(e,[f,f'])\leq\delta\Leftrightarrow d(e,[f,f'])^2-\delta^2\leq 0\Leftrightarrow \brackets{\vec{fe},\vec{fe}}-\brackets{\vec{fh},\vec{fh}}-\delta^2\leq 0$ (we work only with positive values). Then we want: 
\[\begin{array}{r l}
& \forall e\in S, \min_{f,f'\in S}((\brackets{\vec{fe},\vec{fe}}-\brackets{\vec{fh},\vec{fh}}-\delta^2)x_f x_{f'})<0\\
\Leftrightarrow & \forall e\in S,\sum_{f,f'\in S} (\brackets{\vec{fe},\vec{fe}}-\brackets{\vec{fh},\vec{fh}}-\delta^2)x_f x_{f'}<\sum_{f,f'\in S} |\brackets{\vec{fe},\vec{fe}}-\brackets{\vec{fh},\vec{fh}}-\delta^2|x_f x_{f'}\\
\Leftrightarrow & \forall e\in S,\sum_{f,f'\in S} (|\brackets{\vec{fe},\vec{fe}}-\brackets{\vec{fh},\vec{fh}}-\delta^2|-\brackets{\vec{fe},\vec{fe}}+\brackets{\vec{fh},\vec{fh}}+\delta^2)x_f x_{f'}>0
\end{array}\]

We will denote $a_{e,f,f'}=|\brackets{\vec{fe},\vec{fe}}-\brackets{\vec{fh},\vec{fh}}-\delta^2|-\brackets{\vec{fe},\vec{fe}}+\brackets{\vec{fh},\vec{fh}}+\delta^2$.

We now introduce the integer program corresponding to this problem, we call it $(P_{\delta,1})$.
\[\begin{array}{r l}
&\min\sum_{e\in S} x_e\\
\text{respect to}&\forall e\in S, \sum_{f,f'\in S}a_{e,f,f'}x_fx_{f'}>0\\
&\forall e\in S, x_e\in\{0,1\}
\end{array}\]

The major problem is that we have products of variables in the constraint that must not exist. That is way we introduce $z_{f,f'}=x_fx_{f'}$. To do that we just have to put some new constraints, $\forall f,f'\in S, z_{f,f'}\leq x_f \wedge z_{f,f'}\leq x_{f'}\wedge z_{f,f'}\geq \frac{x_f+x_{f'}}{2}-\frac{1}{2} \wedge z_{f,f'}\in\{0,1\}$.

\begin{itemize}
\item if $x_f=x_{f'}=0$, then $z_{f,f'}=0\geq-\frac{1}{2}$.
\item if $x_f=1$ and $x_{f'}=0$ then $z_{f,f'}=0\geq\frac{1}{2}-\frac{1}{2}$.
\item if $x_f=0$ and $x_{f'}=1$ then $z_{f,f'}=0\geq\frac{1}{2}-\frac{1}{2}$.
\item if $x_f=1$ and $x_{f'}=1$ then $z_{f,f'}=1\geq1-\frac{1}{2}$.
\end{itemize}

We now introduce the modified integer program corresponding to this problem, we call it $(P_\delta)$.
\[\begin{array}{r l}
&\min\sum_{e\in S} x_e\\
\text{respect to}&\forall e\in S, \sum_{f,f'\in S}a_{e,f,f'}z_{f,f'}>0\\
&\forall f,f'\in S, z_{f,f'}\leq x_f\\
&\forall f,f'\in S, z_{f,f'}\leq x_{f'}\\
&\forall f,f'\in S, z_{f,f'}\geq \frac{x_f+x_{f'}}{2}-\frac{1}{2}\\
&\forall e\in S, x_e\in\{0,1\}\\
&\forall f,f'\in S, z_{f,f'}\in\{0,1\}
\end{array}\]

This program return a set of point that minimizes the number of points and such that the maximum distance between an element of $S$ and the convex hull of the returned set is at most $\delta$. Then if $S$ represents a convex set, it will be approximated by a polygon which will be included in and the that the border have maximum distance $\delta$. 
\subsection{Minimizing the distance to the polygon for each initial pixel}
We now introduce another program. One can notice that in the previous chapter we can't know many points we will get and depending on $\delta$ it can tends to $|S|$. If one needs to have at most $m$ points, one can consider the new program $(P_m)$:

\end{document}
